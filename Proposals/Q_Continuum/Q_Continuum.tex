%%%%%%%%%%%%%%%%%%%%%%%%%%%%%%%%%%%%%%%%%%%%%%%%%%%%%%%%%%%%%%%%%%%%%%%%%%
%
%    JWST_sci_template.tex  (use only for JWST General Observer and Archival Research proposals)
%
%
%
%    JAMES WEBB SPACE TELESCOPE 
%    OBSERVING PROPOSAL TEMPLATE 
%    FOR CYCLE 1 (2017)
%
%    Version 1.0 September 2017.
%
%    Guidelines and assistance
%    =========================
%     Cycle 1 Announcement Web Page:
%
%         https://jwst-docs.stsci.edu/display/JSP/JWST+Cycle+1+Proposal+Opportunities
%
%    Please contact the JWST Help Desk if you need assistance with any
%    aspect of your proposal:
%    	    http://jwsthelp.stsci.edu
%
%
%
%%%%%%%%%%%%%%%%%%%%%%%%%%%%%%%%%%%%%%%%%%%%%%%%%%%%%%%%%%%%%%%%%%%%%%%%%%%

% The template begins here. Please do not modify the font size from 12 point.

\documentclass[12pt]{article}
\usepackage{jwstproposaltemplate}

\newcommand{\mgii}{Mg\,{\sc ii}\ }

\begin{document}

%   1. SCIENTIFIC JUSTIFICATION
%       (see https://jwst-docs.stsci.edu/jwst-opportunities-and-policies/jwst-call-for-proposals-for-cycle-1/jwst-cycle-1-proposal-preparation)
%
%
\justification          % Do not delete this command.
% Enter your scientific justification here.
%
%
% Page Limit for the Scientific Justification:
%   4: Small GO,     Calibration, and Survey,  AR (Regular, Theory, or Community Data Science Software)
%   5: Medium GO	
%

Actively accreting supermassive black holes, i.e. quasars, are known to exist at very high redshift, when the Universe was
much less than a billion years old. 
However, we currently do not know how these first supermassive black holes (SMBHs) formed.
From estimates of the black hole mass, it is seen that SMBHs with $>10^{9}$ M$_{\odot}$ exist less than a billion years after the Big Bang.
%(usually based on quasar luminosity and the FWHM of the \mgii Mg ii line, under the assumption that local scaling relations are still valid at high luminosity and high redshift.) 
Current models (e.g. Thispaper (2012); Thispaper (2012), Thispaper (2012)) suggestion either
direct collapse black holes (DCBHs) or Population {\sc III} stars could be the progenitors of the very-high redshift SMBHs. 

% So, the General Plan would be::
{\it We propose to obtain MIRI imaging, coronagraphy and spectroscopy of all known $z \geq 6.8$ quasars.} 
%%
%MIRI covers 4.9$\mu$-28.8$\mu$m thus granting access to the optical/near-IR  0.65-3.6$\mu$m rest-frame at $z=6.70$.
MIRI covers 4.9-28.8$\mu$m thus granting access to the optical/near-IR  0.63-3.7$\mu$m rest-frame at $z=6.8$.
%MIRI covers 4.9$\mu$-28.8$\mu$m thus granting access to the optical/near-IR  0.61.-3.6$\mu$m rest-frame at $z=7.00$.
%MIRI covers 4.9$\mu$-28.8$\mu$m thus granting access to the optical/near-IR  0.57-3.37$\mu$m rest-frame at $z=7.54$.
%MIRI covers 4.9$\mu$-28.8$\mu$m thus granting access to the optical/near-IR  0.54-3.2$\mu$m rest-frame at $z=8.00$.
%MIRI covers 4.9$\mu$-28.8$\mu$m thus granting access to the optical/near-IR  0.49-2.88$\mu$m rest-frame at $z=9.00$.
%MIRI covers 4.9$\mu$-28.8$\mu$m thus granting access to the optical/near-IR  0.44-2.62$\mu$m rest-frame at $z=10.00$.

%As a further guide, the Universe is:  850, 816 and 800 Myrs old at z=6.50, z=6.70 and z=6.80, respectively,  for a flat, H0= 69.6, OmegaM = 0.286, OmegaLambda = 0.714 cosmology.

Three ingredients for a quasar: 
{\it (i)} supermassive black hole;  
{\it (ii)} fuel (i.e. gas) supply, annd 
 {\it (iii)} mechanism for getting fuel/gas to the SMBH (i.e., setting up an accretion flow/disk). 

\smallskip
\smallskip
\noindent
The Science Case includes: \\
$\bullet$ How did the first SMBHs form? \\
$\bullet$ What are the BH masses from multiple lines (including rest-frame Balmer)? \\
$\bullet$ What are the host galaxy properties of $z>6.8$ quasars?  \\ %(noting already obtained ALMA data) 
$\bullet$ What is the stellar content of the quasars host galaxy? Does the Magorrian relation hold at very high-$z$? \\
$\bullet$ Is there evidence for (major) merging? \\
$\bullet$ What is the environment of the $z>6.8$ quasars?  \\   %(MIRI imaging FoV is ~0.4 Mpc x 0.6 Mpc on a side at z=7.00)
$\bullet$ Whats the optical/near-IR spectral slope for the highest redshift quasars? \\
$\bullet$ Are the $z>6.8$ luminous quasars different from `regular AGN' detected in  the JWST Deep Fields? \\
$\bullet$ What are the details of the Reionization Epoch? \\



These questions directly address two of the four {\it JWST} science themes
(The End of the Dark Ages: First Light and Reionization; %- JWST will be a powerful time machine with infrared vision that will peer back over 13.5 billion years to see the first stars and galaxies forming out of the darkness of the early universe.
and Assembly of Galaxies) % JWST's unprecedented infrared sensitivity will help astronomers to compare the faintest, earliest galaxies to today's grand spirals and ellipticals, helping us to understand how galaxies assemble over billions of years.
%
No telescope or observatory in the midterm future will be able to access wavelengths longer than
5$\mu$m to these sensitivies. 

Building on the very high redshift quasar catalog in the infrared from
Ross \& Cross (2020, MNRAS accepted) and from Dr. Sarah Bosman’s current list: 
%        http://www.sarahbosman.co.uk/list_of_all_quasars.htm
there are 20 objects with $z \geq 6.80$. 

%If one then takes the Ferruit et al. GTO #1219 proposal as a baseline:
 %       http://www.stsci.edu/jwst/observing-programs/program-information?id=1219
%then the MIRI observations are 4.90 hours charged per object. 
%This comprises of 4.28hrs for MRS spectroscopy and 0.62hours for MIRI F560W imaging. 
%So 29 (20) * 4.90hrs = 142.1 (98.0) hours in total. 
%As a guide, Large programs in Cycle 1 are >75 hours. 

MIRI coronagraphy is a capability unique to JWST.
% so I would strongly suspect  you’d like to use this too. However, I don’t have a great feel yet for the trade-offs of the Lyot-type vs. three 4-quadrant phase-mask (4QPM) coronagraphs.
%% Replicating the full Ferruit et al. GTO #1219 proposal, which includes NIRSpec MSA and IFU  observations as well, is 18.71hrs per object, pushing you to > ~400 hrs (!!) 

Some other key things to note:  
%%
The JWST GTO teams are already looking at $z>6.70$ quasars. 
However, %from what I can tell (see attached JWST_GTO_VHzQ.notes)
only quasar J1342+0928 at $z=7.54$ is currently being observed by MIRI. 
%%
Other teams and collaborations will also be studying the very high-redshifts quasars. 
A key point here is not to directly compete with those observations, but moreover, complement them.
As such, our team has a {\it a zero  proprietary period} making the data immediately accessible from
a webpages (e.g. github.com/d80b2t/JWST\_Cycle1)
%%
The GTO and ERS programs already have the targets and instruments and modes 
that they are going to be using. One could then start to build-up on that arguing that 
you’ll never have the lambda >5um space-based data again any time again soon, 
and
Generating a sample for a large number (all) of the currnently known $z>6.7$ quasars is exactly 
what JWST was built for, and this will tremendously legacy value. 
%%
This could/would be suggested as a multi-cycle program (though not necessarily explicitly 
as that I don’t think this category is implemented yet). Using the Cycle 1+GTO+(ERS?) 
data as type of Pilot program, observing maybe $\sim$5-10 objects in total first, and letting 
that guide observations in future cycles. 
















%%%%%%%%%%%%%%%%%%%%%%%%%%%%%%%%%%%%%%%%%%%%%%%%%%%%%%%%%%%%%%%%%%%%%%%%%%%

%   2. TECHNICAL JUSTIFICATION
%       (see https://jwst-docs.stsci.edu/jwst-opportunities-and-policies/jwst-call-for-proposals-for-cycle-1/jwst-cycle-1-proposal-preparation)
%
%
\justifyobservations   % Do not delete this command.
% Enter your description of the observations.
Describe the overall experimental design of the program, justifying the selection of instruments, modes, exposure times, and requirements. Describe how the observations contribute to the goals described in the scientific justification. Quantitative estimates must be provided of the accuracy required to achieve key science goals. The JWST ETC generally provides sufficient information to determine the necessary exposure time. For modes that require target acquisition, proposers should verify that the exposure specifications provided meet the stated criteria for success. Successful target acquisitions are crucial for the success of the specified observations, and must be verified. The description should also include the following:

\begin{enumerate}
\item Special Observational Requirements (if any): Justify any special scheduling requirements, including time-critical observations. Target of Opportunity observations should estimate the probability of occurrence during Cycle 1, specify whether long-term status is requested, identify whether ToOs are disruptive or non-disruptive, and state clearly how soon JWST must begin observing after the formal activation. 
\item Justification of Coordinated Parallels (if any): Proposals that include coordinated parallel observations should provide a scientific justification for and description of the parallel observations. It should be clearly indicated whether the parallel observations are essential to the interpretation of the primary observations or the science program as a whole, or whether they address partly or completely unrelated issues. The parallel observations are subject to scientific review, and can be rejected even if the primary observations are approved. 
\item Justification of Duplications (if any): as detailed in the JWST Cycle 1 Proposal Policies and Funding Support and the JWST Duplicate Observations Policy. Any duplicate observations must be explicitly justified.
\end{enumerate}

%%%%%%%%%%%%%%%%%%%%%%%%%%%%%%%%%%%%%%%%%%%%%%%%%%%%%%%%%%%%%%%%%%%%%%%%%%%%%%%%
%% 
%%
%%   
%%
%%
%%%%%%%%%%%%%%%%%%%%%%%%%%%%%%%%%%%%%%%%%%%%%%%%%%%%%%%%%%%%%%%%%%%%%%%%%%%%%%%%


\begin{table}
  \begin{tabular}{l  ll l   ccccc}
    \hline \hline
    % Name            & R.A. (J2000) & Decl. (J2000) & R.A. (J2000) & Decl. (J2000)       & redshift & est. $g$-mag \\
    \multirow{2}{*}{Object}  &  {R.A. (J2000)} & Decl. (J2000) & \multirow{2}{*}{redshift} & $K$-band & W1  & W2 & W3 & W4 \\
                                         &     / deg          &  / deg            &                                      & mag           & flux &flux  &  flux & flux \\
    \hline
    J1342+0928  &	205.53375    & +9.47738         &	7.54	     &	                & & & & 	 		    		  \\
    Ponuiaena	  &     215.0	    & 	0.00		    & 7.515    &  	                & & & & 	 		    		  \\	 	
    J1120+0641  & 	170.00616   & 	6.69008        &	7.0842	  &  	                & & & & 	 \\
                                                   \hline       
                                                    \hline
\end{tabular}
\caption{Very high-$z$ quasars.
  
}
     \label{tab:previous_surveys}
%  \end{center}
\end{table}
\normalsize


%\end{document}



  
%%%%%%%%%%%%%%%%%%%%%%%%%%%%%%%%%%%%%%%%%%%%%%%%%%%%%%%%%%%%%%%%%%%%%%%%%%%

%   3. SPECIAL REQUIREMENTS
%        (see https://jwst-docs.stsci.edu/jwst-opportunities-and-policies/jwst-call-for-proposals-for-cycle-1/jwst-cycle-1-proposal-preparation)
%
%
\specialreq             % Do not delete this command.
% Justify your special requirements here, if any.

There are no Special Requirements. 


%%%%%%%%%%%%%%%%%%%%%%%%%%%%%%%%%%%%%%%%%%%%%%%%%%%%%%%%%%%%%%%%%%%%%%%%%%%

%   4. COORDINATED PARALLEL OBSERVATIONS
%        (see https://jwst-docs.stsci.edu/jwst-opportunities-and-policies/jwst-call-for-proposals-for-cycle-1/jwst-cycle-1-proposal-preparation)
%
%
\coordinatedobs % Do not delete this command.
% Enter your coordinated parallel observing plans here, if any.

1	NIRCam imaging$^{*}$ MIRI imaging$^{*}$ Either template can be selected as primary, with the other as parallel. \\
2	NIRCam imaging$^{*}$	NIRISS WFSS	Either template can be selected as primary, with the other as parallel. \\ 
3	MIRI imaging	NIRISS WFSS	Either template can be selected as primary, with the other as parallel. \\
4	NIRCam imaging$^{*}$ NIRISS imaging	NIRCam must be primary. Use to increase areal coverage, but note NIRISS differences in pixel size and available filters. \\
5	NIRSpec MOS	NIRCam imaging	NIRSpec MOS must be primary.\\
(Modes added January 2020)::	
6	NIRCam WFSS	MIRI Imaging	NIRCam WFSS must be primary. \\
7	NIRCam WFSS	NIRISS Imaging	NIRCam WFSS must be primary.\\
8	NIRSpec MOS	MIRI Imaging	NIRSpec MOS must be primary. \\

$^{*}$Only direct imaging with standard narrow-, medium-, or broadband filters is allowed for NIRCam and MIRI observations in these coordinated parallel modes.


%%%%%%%%%%%%%%%%%%%%%%%%%%%%%%%%%%%%%%%%%%%%%%%%%%%%%%%%%%%%%%%%%%%%%%%%%%%

%   5. JUSTIFY DUPLICATIONS
%        (see https://jwst-docs.stsci.edu/jwst-opportunities-and-policies/jwst-call-for-proposals-for-cycle-1/jwst-cycle-1-proposal-preparation)
%
%
\duplications           % Do not delete this command.
% Enter your duplication justifications here, if any.




%%%%%%%%%%%%%%%%%%%%%%%%%%%%%%%%%%%%%%%%%%%%%%%%%%%%%%%%%%%%%%%%%%%%%%%%%%%

%   6. ANALYSIS PLAN
%       (see https://jwst-docs.stsci.edu/jwst-opportunities-and-policies/jwst-call-for-proposals-for-cycle-1/jwst-cycle-1-proposal-preparation)
%
%
\analysisplan % Do not delete this command.
% Describe the data processing and analysis plan here.

%%%%%%%%%%%%%%%%%%%%%%%%%%%%%%%%%%%%%%%%%%%%%%%%%%%%%%%%%%%%%%%%%%%%%%%%%%%
Analysis Plan:  {\it (required only for AR, Calibration, and Theory Proposals)} All AR Proposals should provide a detailed data analysis plan and describe the datasets that will be analyzed. Inclusion of a target list is not required.

\smallskip
\smallskip
\noindent
Legacy AR Proposals should also discuss the data products that will be made available to the community, the method of dissemination, and a realistic timeline. It is a requirement that data products be delivered to STScI in suitable digital formats for further dissemination via the MAST Data Archive or related channels. Any required technical support from STScI and associated costs should be described in detail.

\smallskip
\smallskip
\noindent
Theory Proposals should discuss the types of JWST data that will benefit from the proposed investigation, and references to specific data sets in the MAST Data Archive should be given where possible. They should also describe how the results of the theoretical investigation will be made available to the astronomical community, and on what timescale the results are expected.

\smallskip
\smallskip
\noindent
Calibration Proposals should discuss what documentation, and data products and/or software will be made available to STScI to support future observing programs. Proposers should explain how their programs complement ongoing calibration efforts by the STScI instrument groups. They should contact the relevant instrument groups to ensure that efforts are not duplicated, and if they are, justify why the duplications are necessary.

\smallskip
\smallskip
\noindent
During the budget review process, the Financial Review Committee will compare the requested costs with the commensurate work outlined in the Analysis Plan. Support for resources outside the original scope of work will not be considered.



\end{document}          % End of proposal. Do not delete this line.
                        % Everything after this command is ignored.


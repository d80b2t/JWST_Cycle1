\documentclass[11pt]{article}

\usepackage{natbib}
\usepackage{hyperref}


%%%%%%%%%%%%%%%%%%%%%%%%%%%%%%%%%%%%%%%%%%%
%       define Journal abbreviations      %
%%%%%%%%%%%%%%%%%%%%%%%%%%%%%%%%%%%%%%%%%%%
\def\nat{Nat} \def\apjl{ApJ~Lett.} \def\apj{ApJ}
\def\apjs{ApJS} \def\aj{AJ} \def\mnras{MNRAS}
\def\prd{Phys.~Rev.~D} \def\prl{Phys.~Rev.~Lett.}
\def\plb{Phys.~Lett.~B} \def\jhep{JHEP}
\def\npbps{NUC.~Phys.~B~Proc.~Suppl.} \def\prep{Phys.~Rep.}
\def\pasp{PASP} \def\aap{Astron.~\&~Astrophys.} \def\araa{ARA\&A}
\newcommand{\preep}[1]{{\tt #1} }



\begin{document}

\title{Some general Very high-$z$ quasar and JWST Science notes} 
\author{Nicholas P. Ross}
\date{\today}
\maketitle


\section*{Salpeter time}
The Salpeter timescale (or Salpeter time) is the timescale for black
hole growth, based upon the Eddington limit: a growing black hole
heats accretion material, which glows and is subject to the luminosity
limit. The timescale is $\approx 5 \times 10^{7}$ years. 



\begin{equation}
  t_{\rm Sal} = M/\dot{M} = 4.5 \times10^{7} \left( \frac{\epsilon}{0.1}\right ) \left( \frac{L}{L_{\rm Edd}}\right )^{-1}
\end{equation}
where $\epsilon = L/\dot{M} c^{2}$ is the radiative efficiency for a
QSO radiating at a fraction $L/L_{\rm Edd}$ of the Eddington
luminosity. Commonly accepted values of these two key parameters for
luminous QSOs are $\epsilon= 0.1$ and $L/LEdd = 1$.  Martini, P. (QSO
Lifetimes; http://adsabs.harvard.edu/abs/2004cbhg.symp..169M).

This critical accretion rate, [the Eddington mass accretion rate], is
proportional to the mass of the accreting object, which implies that
the mass of an object that is growing at the maximal (Eddington)
accretion grows exponentially on a timescale known as the Salpeter
time,
\begin{eqnarray}
  t_{\rm Sal} & = &  \frac{\epsilon \sigma_{T} c}{4\pi G m_{p}}  \\
              & = & \frac{\epsilon \cdot 6.65\times10^{-29}     \cdot           3\times10^{8}}
                    {4\pi       \cdot  6.674\times10^{-11}  \cdot  1.6726\times10^{-27} }  \;\; {\rm seconds}\\
              & = & 1.4 \epsilon \times 10^{16} \;\; {\rm seconds}\\
               & \approx & 45 \epsilon \times  10^7 \, {\rm years} \\
                 & \approx & 45 \epsilon_{0.1}\times  10^6 \, {\rm years}
\end{eqnarray}
where $\epsilon_{0.1}$ is the efficiency at which accreting
gas rest mass energy is converted in radiation expressed in units of
10\%, the standard value for an accreting Schwarzchild black hole. (In
terms of this efficiency, the accretion luminosity is $L_{\rm acc} =
\epsilon \dot{M}_{rm BH} c^{2}$, where $\dot{M}_{\rm BH}$ is the black hole
accretion rate.)
From \citep{Coppi2003}.
%`Massive Black Hole Growth and Formation'' Paolo Coppi.

%%tSal = εσT c ≈ 45ε0.1106 years, (1) 4πGmp

\citet{Salpeter64}.


\clearpage
\bibliographystyle{mn2e}
\bibliography{/cos_pc19a_npr/LaTeX/tester_mnras}

\end{document}

